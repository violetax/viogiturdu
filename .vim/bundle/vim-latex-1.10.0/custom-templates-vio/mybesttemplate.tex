<+  +>  !comp!  !exe!
%previously proyectoinv.tex
%%up until "\begin{document}" is the "header." This sets up the whole document (format, style, additional things)
%\command[modifier1, modifier2]{argument} is a general format
\documentclass[12pt, amssymb, one column]{article}
\usepackage[utf8]{inputenc}
\usepackage{times, cancel, changepage, graphicx, fancyhdr, graphicx, enumitem, hyperref}
\def\UrlBreaks{\do\/\do-}
\hypersetup{
colorlinks=true,
linkcolor=blue,
filecolor=magenta,
urlcolor=cyan,
}
\urlstyle{same}
\usepackage[hyphenbreaks]{breakurl}
\usepackage[hyphens]{url}
\graphicspath{ {./images} }
%\usepackage[margin=1.5in]{geometry}
\usepackage[
  height=9in,      % height of the text block
  width=7in,       % width of the text block
  top=78pt,        % distance of the text block from the top of the page
  headheight=48pt, % height for the header block
  headsep=12pt,    % distance from the header block to the text block
  heightrounded,   % ensure an integer number of lines
%  showframe,       % show the main blocks
  verbose,         % show the values of the parameters in the log file
]{geometry}
\usepackage{fancyhdr}

\newcommand{\ihat}{\mathbf {\hat \imath}}
\newcommand{\jhat}{\mathbf {\hat \jmath}}

%you only really need a documentclass/ packages to make a document, the following is just nice formatting. You can easily remove it or look up something else
%that you like better

\pagestyle{fancy}

\fancyhead[L]{Violeta González}
\fancyhead[C]{\huge %%%  \normalsize}
\fancyhead[R]{%%% \\%%%}
\fancypagestyle{plain}{
\fancyhead[LE, RO]{Violeta González}
\fancyhead[LO, RE]{%%%}}

\usepackage{setspace}
\renewcommand{\baselinestretch}{1.5}

%end of header, beginning of document

\begin{document}

%To insert image:
%\begin{center}
%\includegraphics[scale=.8]{ProblemSet3Graph2.png}
%\end{center}
%Note that the file must be in the same folder. 




\end{document}
