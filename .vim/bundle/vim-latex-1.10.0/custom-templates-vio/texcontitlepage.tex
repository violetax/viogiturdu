<+  +>  !comp!  !exe!
%\command[modifier1, modifier2]{argument} is a general format
\documentclass[12pt, amssymb, one column]{article}
\usepackage[utf8]{inputenc}
\usepackage{biblatex}
\usepackage{times, graphicx, fancyhdr, graphicx, enumitem, hyperref}
\def\UrlBreaks{\do\/\do-}
\hypersetup{
  colorlinks=true,
  linkcolor=blue,
  filecolor=magenta,
  urlcolor=cyan,
}
\urlstyle{same}
\usepackage[hyphenbreaks]{breakurl}
\usepackage[hyphens]{url}
\graphicspath{ {./images} }
%\usepackage[margin=1.5in]{geometry}
\usepackage[
    height=9in,      % height of the text block
  width=7in,       % width of the text block
    top=78pt,        % distance of the text block from the top of the page
  headheight=48pt, % height for the header block
    headsep=12pt,    % distance from the header block to the text block
  heightrounded,   % ensure an integer number of lines
  %  showframe,       % show the main blocks
    verbose,         % show the values of the parameters in the log file
  ]{geometry}
  \usepackage{fancyhdr}
  \usepackage{tocloft}
  \pagestyle{fancy}

\fancyhead[L]{Violeta González}
\fancyhead[R]{ \normalsize}
%\fancyhead[R]{Investigación científica}
\fancypagestyle{plain}{
\fancyhead[LO, RE]{Violeta González}
\fancyhead[LE, RO]{Universidad de León\\ASIGNATURA}}


\DefineBibliographyStrings{english}{%
    references = {},
  }

\renewcommand*\contentsname{Contenidos}
%end of header, beginning of document

% Keywords command
\providecommand{\keywords}[1]
{
    \small	
	\textbf{\textit{Keywords---}} #1
  }

\begin{document}

%To insert image:
%\begin{center}
%\includegraphics[scale=.8]{ProblemSet3Graph2.png}
%\end{center}
%Note that the file must be in the same folder. 

\begin{titlepage}
 \begin{center}
           \vspace*{1cm}
    
		 \textbf{%%%}
   
         \vspace{0.5cm}
        %%% 
        \vspace{1.5cm}
     
           \textbf{Violeta González García}
    
           \vspace{0.8cm}
    



   
     \end{center}
   \end{titlepage}


\tableofcontents
\newpage

%%% %START HERE!

\section{Fuentes de información y referencias bibliográficas}
\subsection{Análisis de la bibliografía}
\subsection{Referencias}
\begin{thebibliography}{}
  \bibitem{1}
	%SURNAME, %name, et. al, año. %Title. En: \textit{%where} [en línea]. Lugar de publicación: %where. Consulta: \today. Disponible en: \burl{%url}
\end{thebibliography}

	
	\end{document}
